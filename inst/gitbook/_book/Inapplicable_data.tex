\documentclass[]{book}
\usepackage{lmodern}
\usepackage{amssymb,amsmath}
\usepackage{ifxetex,ifluatex}
\usepackage{fixltx2e} % provides \textsubscript
\ifnum 0\ifxetex 1\fi\ifluatex 1\fi=0 % if pdftex
  \usepackage[T1]{fontenc}
  \usepackage[utf8]{inputenc}
\else % if luatex or xelatex
  \ifxetex
    \usepackage{mathspec}
  \else
    \usepackage{fontspec}
  \fi
  \defaultfontfeatures{Ligatures=TeX,Scale=MatchLowercase}
\fi
% use upquote if available, for straight quotes in verbatim environments
\IfFileExists{upquote.sty}{\usepackage{upquote}}{}
% use microtype if available
\IfFileExists{microtype.sty}{%
\usepackage{microtype}
\UseMicrotypeSet[protrusion]{basicmath} % disable protrusion for tt fonts
}{}
\usepackage[margin=1in]{geometry}
\usepackage{hyperref}
\hypersetup{unicode=true,
            pdftitle={Inapplicable data},
            pdfauthor={Martin Brazeau (m.brazeau@imperial.ac.uk), Thomas Guillerme (guillert@tcd.ie) and Martin Smith (martin.smith@durham.ac.uk)},
            pdfborder={0 0 0},
            breaklinks=true}
\urlstyle{same}  % don't use monospace font for urls
\usepackage{natbib}
\bibliographystyle{plainnat}
\usepackage{longtable,booktabs}
\usepackage{graphicx,grffile}
\makeatletter
\def\maxwidth{\ifdim\Gin@nat@width>\linewidth\linewidth\else\Gin@nat@width\fi}
\def\maxheight{\ifdim\Gin@nat@height>\textheight\textheight\else\Gin@nat@height\fi}
\makeatother
% Scale images if necessary, so that they will not overflow the page
% margins by default, and it is still possible to overwrite the defaults
% using explicit options in \includegraphics[width, height, ...]{}
\setkeys{Gin}{width=\maxwidth,height=\maxheight,keepaspectratio}
\IfFileExists{parskip.sty}{%
\usepackage{parskip}
}{% else
\setlength{\parindent}{0pt}
\setlength{\parskip}{6pt plus 2pt minus 1pt}
}
\setlength{\emergencystretch}{3em}  % prevent overfull lines
\providecommand{\tightlist}{%
  \setlength{\itemsep}{0pt}\setlength{\parskip}{0pt}}
\setcounter{secnumdepth}{5}
% Redefines (sub)paragraphs to behave more like sections
\ifx\paragraph\undefined\else
\let\oldparagraph\paragraph
\renewcommand{\paragraph}[1]{\oldparagraph{#1}\mbox{}}
\fi
\ifx\subparagraph\undefined\else
\let\oldsubparagraph\subparagraph
\renewcommand{\subparagraph}[1]{\oldsubparagraph{#1}\mbox{}}
\fi

%%% Use protect on footnotes to avoid problems with footnotes in titles
\let\rmarkdownfootnote\footnote%
\def\footnote{\protect\rmarkdownfootnote}

%%% Change title format to be more compact
\usepackage{titling}

% Create subtitle command for use in maketitle
\newcommand{\subtitle}[1]{
  \posttitle{
    \begin{center}\large#1\end{center}
    }
}

\setlength{\droptitle}{-2em}
  \title{Inapplicable data}
  \pretitle{\vspace{\droptitle}\centering\huge}
  \posttitle{\par}
  \author{Martin Brazeau
(\href{mailto:m.brazeau@imperial.ac.uk}{\nolinkurl{m.brazeau@imperial.ac.uk}}),
Thomas Guillerme
(\href{mailto:guillert@tcd.ie}{\nolinkurl{guillert@tcd.ie}}) and Martin
Smith
(\href{mailto:martin.smith@durham.ac.uk}{\nolinkurl{martin.smith@durham.ac.uk}})}
  \preauthor{\centering\large\emph}
  \postauthor{\par}
  \predate{\centering\large\emph}
  \postdate{\par}
  \date{2018-03-07}

\usepackage{booktabs}

\usepackage{amsthm}
\newtheorem{theorem}{Theorem}[chapter]
\newtheorem{lemma}{Lemma}[chapter]
\theoremstyle{definition}
\newtheorem{definition}{Definition}[chapter]
\newtheorem{corollary}{Corollary}[chapter]
\newtheorem{proposition}{Proposition}[chapter]
\theoremstyle{definition}
\newtheorem{example}{Example}[chapter]
\theoremstyle{definition}
\newtheorem{exercise}{Exercise}[chapter]
\theoremstyle{remark}
\newtheorem*{remark}{Remark}
\newtheorem*{solution}{Solution}
\begin{document}
\maketitle

{
\setcounter{tocdepth}{1}
\tableofcontents
}
\hypertarget{inapplicable-data-in-a-parsimony-setting}{%
\chapter*{Inapplicable data in a parsimony
setting}\label{inapplicable-data-in-a-parsimony-setting}}
\addcontentsline{toc}{chapter}{Inapplicable data in a parsimony setting}

This document provides a detailed explanation of the algorithm for
handling inapplicable data proposed by Brazeau \emph{et al.}
\citeyearpar{ThisStudy}.

We first discuss how the \href{../fitch}{Fitch algorithm} works and
introduce the \url{problems} that it encounters in the face of
inapplicable character states.

We then introduce our \url{solution}, a new \url{algorithm}, implemented
in various \href{software}{software packages}, and discuss its
implications for the coding of \href{coding}{characters} and
\url{ambiguity}.

We close with some \href{examples}{example} trees that demonstrate how
our algorithm behaves in more complicated cases.

\hypertarget{fitch}{%
\chapter{The Fitch algorithm}\label{fitch}}

Placeholder

\hypertarget{a}{%
\section{a}\label{a}}

\hypertarget{b}{%
\section{b}\label{b}}

\hypertarget{c}{%
\section{c}\label{c}}

\hypertarget{d}{%
\section{d}\label{d}}

\hypertarget{e}{%
\section{e}\label{e}}

\hypertarget{n2}{%
\section{n2}\label{n2}}

\hypertarget{n1}{%
\section{n1}\label{n1}}

\hypertarget{n3}{%
\section{n3}\label{n3}}

\hypertarget{n4}{%
\section{n4}\label{n4}}

\hypertarget{first-cherry}{%
\section{First cherry}\label{first-cherry}}

\hypertarget{second-cherry}{%
\section{Second cherry}\label{second-cherry}}

\hypertarget{third}{%
\section{Third}\label{third}}

\hypertarget{fourth}{%
\section{Fourth}\label{fourth}}

\hypertarget{first-cherry-1}{%
\section{First cherry}\label{first-cherry-1}}

\hypertarget{second-cherry-1}{%
\section{Second cherry}\label{second-cherry-1}}

\hypertarget{third-1}{%
\section{Third}\label{third-1}}

\hypertarget{fourth-1}{%
\section{Fourth}\label{fourth-1}}

\hypertarget{plotting-the-tree-and-characters}{%
\section{Plotting the tree and
characters}\label{plotting-the-tree-and-characters}}

\hypertarget{loading-the-inapp-package}{%
\section{Loading the Inapp package}\label{loading-the-inapp-package}}

\hypertarget{the-tree}{%
\section{The tree}\label{the-tree}}

\hypertarget{the-character}{%
\section{The character}\label{the-character}}

\hypertarget{applying-the-fitch-algorithm}{%
\section{Applying the Fitch
algorithm}\label{applying-the-fitch-algorithm}}

\hypertarget{downpass}{%
\section{Downpass}\label{downpass}}

\hypertarget{plotting-the-first-downpass}{%
\section{Plotting the first
downpass}\label{plotting-the-first-downpass}}

\hypertarget{uppass}{%
\section{Uppass}\label{uppass}}

\hypertarget{plotting-the-first-downpass-1}{%
\section{Plotting the first
downpass}\label{plotting-the-first-downpass-1}}

\hypertarget{running-the-inapp-app}{%
\section{Running the Inapp App}\label{running-the-inapp-app}}

\hypertarget{resolving-ambiguous-resolutions}{%
\section{Resolving ambiguous
resolutions}\label{resolving-ambiguous-resolutions}}

\hypertarget{problems}{%
\chapter{Problems with the Fitch algorithm}\label{problems}}

Placeholder

\hypertarget{red-tails-blue-tails}{%
\section{Red tails, blue tails}\label{red-tails-blue-tails}}

\hypertarget{why-reductive-coding-doesnt-work}{%
\section{Why Reductive coding doesn't
work}\label{why-reductive-coding-doesnt-work}}

\hypertarget{an-exception}{%
\subsection{An exception}\label{an-exception}}

\hypertarget{why-extra-state-coding-doesnt-work}{%
\section{Why Extra State coding doesn't
work}\label{why-extra-state-coding-doesnt-work}}

\hypertarget{why-a-single-multi-state-character-doesnt-work}{%
\section{Why a single multi-state character doesn't
work}\label{why-a-single-multi-state-character-doesnt-work}}

\hypertarget{sankoff-matrices}{%
\section{Sankoff matrices}\label{sankoff-matrices}}

\hypertarget{symmetric}{%
\subsection{Symmetric}\label{symmetric}}

\hypertarget{gain-and-loss-asymmetric}{%
\subsection{Gain and loss asymmetric}\label{gain-and-loss-asymmetric}}

\hypertarget{why-counting-steps-cannot-work}{%
\section{Why counting steps cannot
work}\label{why-counting-steps-cannot-work}}

\hypertarget{conclusion}{%
\section{Conclusion}\label{conclusion}}

\hypertarget{solution}{%
\chapter{A solution}\label{solution}}

Placeholder

\hypertarget{minimising-homoplasy}{%
\section{Minimising homoplasy}\label{minimising-homoplasy}}

\hypertarget{what-does-it-take-to-denote-separate-regions}{%
\subsection{What does it take to denote separate
regions?}\label{what-does-it-take-to-denote-separate-regions}}

\hypertarget{how-this-fixes-the-problem}{%
\subsection{How this fixes the
problem}\label{how-this-fixes-the-problem}}

\hypertarget{summary}{%
\subsection{Summary}\label{summary}}

\hypertarget{algorithm}{%
\section{Algorithmic implementation}\label{algorithm}}

\hypertarget{plotting-the-tree}{%
\section{Plotting the tree}\label{plotting-the-tree}}

\hypertarget{loading-the-inapp-package-1}{%
\section{Loading the Inapp package}\label{loading-the-inapp-package-1}}

\hypertarget{the-tree-1}{%
\section{The tree}\label{the-tree-1}}

\hypertarget{the-character-1}{%
\section{The character}\label{the-character-1}}

\hypertarget{applying-the-na-algorithm}{%
\section{Applying the NA algorithm}\label{applying-the-na-algorithm}}

\hypertarget{passes-1-2}{%
\subsection{Passes 1 \& 2}\label{passes-1-2}}

\hypertarget{plotting-the-na-two-first-passes}{%
\section{Plotting the NA two first
passes}\label{plotting-the-na-two-first-passes}}

\hypertarget{the-parent-character}{%
\section{The parent character}\label{the-parent-character}}

\hypertarget{applying-the-fitch-algorithm-1}{%
\section{Applying the Fitch
algorithm}\label{applying-the-fitch-algorithm-1}}

\hypertarget{pass-3}{%
\subsection{Pass 3}\label{pass-3}}

\hypertarget{plotting-the-na-two-first-passes-1}{%
\section{Plotting the NA two first
passes}\label{plotting-the-na-two-first-passes-1}}

\hypertarget{tracking-applicable-regions}{%
\subsubsection{Tracking applicable
regions}\label{tracking-applicable-regions}}

\hypertarget{plotting-the-na-two-first-passes-2}{%
\section{Plotting the NA two first
passes}\label{plotting-the-na-two-first-passes-2}}

\hypertarget{plotting-the-na-two-first-passes-3}{%
\section{Plotting the NA two first
passes}\label{plotting-the-na-two-first-passes-3}}

\hypertarget{pass-4}{%
\subsection{Pass 4}\label{pass-4}}

\hypertarget{plotting-the-na-two-first-passes-4}{%
\section{Plotting the NA two first
passes}\label{plotting-the-na-two-first-passes-4}}

\hypertarget{software}{%
\section{Software implementation}\label{software}}

\hypertarget{coding}{%
\chapter{Coding data}\label{coding}}

Placeholder

\hypertarget{multiple-dependencies}{%
\section{Multiple dependencies}\label{multiple-dependencies}}

\hypertarget{invariant-characters-can-inform-parsimony}{%
\section{Invariant characters can inform
parsimony}\label{invariant-characters-can-inform-parsimony}}

\hypertarget{variable-but-parsimony-uninformative-characters-can-inform-parsimony}{%
\section{Variable but `parsimony uninformative' characters can inform
parsimony}\label{variable-but-parsimony-uninformative-characters-can-inform-parsimony}}

\hypertarget{this-may-not-be-desirable-in-neomorphic-characters}{%
\section{This may not be desirable in neomorphic
characters}\label{this-may-not-be-desirable-in-neomorphic-characters}}

\hypertarget{three-scenarios}{%
\subsection{Three scenarios}\label{three-scenarios}}

\hypertarget{evaluation}{%
\subsubsection{Evaluation}\label{evaluation}}

\hypertarget{does-absence-contain-phylogenetic-information}{%
\subsection{Does absence contain phylogenetic
information?}\label{does-absence-contain-phylogenetic-information}}

\hypertarget{neomorphics}{%
\section{Coding ontologically dependent neomorphic
characters}\label{neomorphics}}

\hypertarget{neomorphic-and-transformational-characters}{%
\subsection{Neomorphic and transformational
characters}\label{neomorphic-and-transformational-characters}}

\hypertarget{three-trees}{%
\subsection{Three trees}\label{three-trees}}

\hypertarget{one-tail-spotted-one-not}{%
\subsubsection{One tail spotted, one
not}\label{one-tail-spotted-one-not}}

\hypertarget{two-spot-appearances}{%
\subsubsection{Two spot appearances}\label{two-spot-appearances}}

\hypertarget{two-non-spotted-appearances}{%
\subsubsection{Two non-spotted
appearances}\label{two-non-spotted-appearances}}

\hypertarget{evaluation-1}{%
\subsubsection{Evaluation}\label{evaluation-1}}

\hypertarget{implications}{%
\subsubsection{Implications}\label{implications}}

\hypertarget{recommendation}{%
\subsection{Recommendation}\label{recommendation}}

\hypertarget{ambiguity}{%
\chapter{Coding ambiguity}\label{ambiguity}}

Placeholder

\hypertarget{principal-character-ambiguous}{%
\section{Principal character
ambiguous}\label{principal-character-ambiguous}}

\hypertarget{principal-character-known}{%
\section{Principal character known}\label{principal-character-known}}

\hypertarget{subordinate-character-has-finite-states}{%
\subsection{Subordinate character has finite
states}\label{subordinate-character-has-finite-states}}

\hypertarget{subordinate-character-may-have-unobserved-states}{%
\subsection{Subordinate character may have unobserved
states}\label{subordinate-character-may-have-unobserved-states}}

\hypertarget{recommendation-1}{%
\section{Recommendation}\label{recommendation-1}}

\hypertarget{examples}{%
\chapter{Examples}\label{examples}}

Placeholder

\hypertarget{some-caterpillars}{%
\section{Some caterpillars}\label{some-caterpillars}}

\hypertarget{three-equally-suboptimal-alternatives}{%
\section{Three equally suboptimal
alternatives}\label{three-equally-suboptimal-alternatives}}

\hypertarget{a-better-caterpillar-tree}{%
\section{A better caterpillar tree}\label{a-better-caterpillar-tree}}

\hypertarget{de-laets-caterpillars}{%
\section{De Laet's caterpillars}\label{de-laets-caterpillars}}

\bibliography{../References.bib,packages.bib}


\end{document}
