\documentclass[a4paper,11pt]{article}

\usepackage{enumerate}
\usepackage[osf]{mathpazo}
\usepackage{lastpage}
\usepackage{url}
\usepackage{hyperref}
\pagenumbering{arabic}
\linespread{1.66}

\begin{document}

\begin{flushright}
Version dated: \today
\end{flushright}
\begin{center}

%Title
\noindent{\Large{\bf{Counting difference}}}\\
\bigskip

%Author
Thomas Guillerme (\href{mailto:t.guillerme@imperial.ac.uk}{t.guillerme@imperial.ac.uk})

\end{center}

\section{Collecting the data}
We collected 37 discrete morphological matrices containing inapplicable data.

We selected these matrices to represent an heterogeneous sampling of matrices (@@@ taxonomic classes).


\section{Inferring the topologies}
We ran the matrices in PAUP* (v 4.0a151).

We arbitrarily set the first taxa in the matrix as the outgroup and ran heursitic searches with random sequences addition replicated 100 times with a maximum of $5 \times 10^6$ rearrangements per replicates. \texttt{hsearch addseq=random nreps=100 rearrlimit=5000000 limitperrep=yes;}

We then saved the tree island with the most parsimonious score treating inapplicable tokens as ambiguous (``?'' - default option in PAUP \texttt{pset gapmode=missing}).

We then counted the tree length of the most parsimonious island treating inapplicable tokens as an extra state (\texttt{pset gapmode=newstate} option in PAUP) or using our new algorithm.

We ran the tree counting using a portable executable using the \texttt{MorphyLib} and \texttt{TreeLib} C libraries (see supplementary).


\section{Comparing the algorithms' effects}
We corrected the tree length under each algorithm by using a relative score defined as:


\begin{equation}
    \text{relative score}=\frac{\text{New score}}{\text{Most parsimonious score}}-1
\end{equation}

Where the new score is the tree length under either treating inapplicable tokens as a new character (hereafter: the ``relative new-state score'') or treating inapplicable tokens under our new algorithm (hereafter the ``relative inapplicable score'').

We compared the distribution of the relative scores of all the trees of the most parsimonious of both islands using the Bhattacharrya Coefficient (probability of overlap between both distributions) and a Wilcoxon ranked mean comparison.



 


\end{document}
